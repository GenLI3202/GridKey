\documentclass[11pt, a4paper]{article}

% --- UNIVERSAL PREAMBLE BLOCK ---
\usepackage[a4paper, top=2.5cm, bottom=2.5cm, left=2cm, right=2cm]{geometry}
\usepackage{fontspec}
\usepackage[english, bidi=basic, provide=*]{babel}
\babelprovide[import, onchar=ids fonts]{english}
\setmainfont{Latin Modern Roman}
\setsansfont{Latin Modern Sans}
\setmonofont{Latin Modern Mono}

% --- DOCUMENT-SPECIFIC PACKAGES ---
\usepackage{amsmath}
\usepackage{amssymb}
\usepackage{booktabs}
\usepackage{graphicx}
\usepackage{caption}
\usepackage{fancyhdr}
\usepackage[
    colorlinks=true,
    linkcolor=blue,
    urlcolor=blue,
    citecolor=blue
]{hyperref}

% --- PAGE STYLE ---
\pagestyle{fancy}
\fancyhf{}
\fancyhead[L]{Huawei TechArena 2025: Phase II EMS Model}
\fancyhead[R]{Gen Li (Team SoloGen)}
\fancyfoot[C]{\thepage}
\renewcommand{\headrulewidth}{0.4pt}
\renewcommand{\footrulewidth}{0.4pt}

% --- TITLE ---
\title{Huawei TechArena 2025: BESS Energy Management System \\ \large Phase II Investment Optimization Model}
\author{Gen Li (Team SoloGen)}
\date{\today}

% --- DOCUMENT START ---
\begin{document}

\maketitle
\thispagestyle{fancy}

\begin{abstract}
This document outlines the mathematical and conceptual framework for the Phase II investment optimization task. The primary goal is to determine the optimal BESS configuration (Energy Capacity and Power Rating) that maximizes the 10-year Net Present Value (NPV) of the project. This model serves as the high-level, strategic decision-making layer, which informs the operational models detailed in other documents. The formulation explicitly incorporates the time value of money through a discount rate (WACC) and accounts for long-term revenue generation, costs, and battery degradation over the fixed 10-year project horizon, drawing insights from Collath et al. (2023).
\end{abstract}

\section{Introduction}
The investment optimization task seeks to answer the fundamental question: "What is the most profitable BESS to build?" This involves a trade-off between the initial capital expenditure (CAPEX) and the potential revenue streams over the asset's lifetime. A larger battery may generate more revenue but costs more upfront and may degrade differently.

This model is formulated as a Net Present Value (NPV) maximization problem over a 10-year period. The NPV framework is standard for capital budgeting projects as it correctly accounts for the time value of money, ensuring that future cash flows are appropriately valued in today's terms.

The annual revenues, which are a key input to this model, are generated by the detailed operational optimization model (as described in `p2\_bi\_model\_ggdp.tex`). That operational model simulates the BESS performance in the DA, FCR, and aFRR markets. The investment model must therefore abstract or estimate these annual revenues as a function of the chosen investment configuration and the battery's state of health (SOH) in any given year.

\section{Investment Optimization Model Formulation}

\subsection{Objective Function: Maximize 10-Year NPV}
The objective is to maximize the Net Present Value of all cash flows over the 10-year project lifetime. The NPV is the sum of discounted future net cash flows minus the initial investment.

\begin{equation}
\max_{\mathbf{x} \in X} \quad \text{NPV}(\mathbf{x}) = -C_{\text{inv}}(\mathbf{x}) + \sum_{y=1}^{10} \frac{\mathbb{E}[\Pi_y(\mathbf{x}, SOH_y)] - C_{\text{O\&M},y}(\mathbf{x})}{(1 + r)^y}
\end{equation}

Where:
\begin{itemize}
    \item $\mathbf{x}$ is the vector of investment decision variables (e.g., $E_{\text{nom}}, P_{\text{max}}$).
    \item $X$ is the set of feasible investment choices.
    \item $C_{\text{inv}}(\mathbf{x})$ is the initial investment cost (CAPEX).
    \item $\mathbb{E}[\Pi_y(\mathbf{x}, SOH_y)]$ is the expected net revenue in year $y$. This is a function of the investment $\mathbf{x}$ and the battery's State of Health ($SOH_y$) at the start of year $y$.
    \item $C_{\text{O\&M},y}(\mathbf{x})$ is the Operations and Maintenance cost for year $y$.
    \item $r$ is the annual discount rate (WACC).
    \item $SOH_y$ is the State of Health at the beginning of year $y$, which evolves over time.
\end{itemize}

\subsection{Model Components}

\subsubsection{Sets and Indices}
\begin{tabular}{@{}ll}
\toprule
$Y, y$ & Set and index for years in the project horizon, $y \in \{1, ..., 10\}$ \\
\bottomrule
\end{tabular}

\subsubsection{Decision Variables}
The primary strategic decisions for the investment model are the BESS physical characteristics.
\begin{tabular}{@{}lll}
\toprule
Symbol & Description & Unit \\
\midrule
$E_{\text{nom}}$ & Nominal energy capacity to be installed & kWh \\
$P_{\text{max}}$ & Maximum power capacity to be installed & kW \\
\bottomrule
\end{tabular}
\textit{Note: These variables are linked by the C-rate: $P_{\text{max}} = C_{\text{rate}} \cdot E_{\text{nom}}$. The optimization can be framed as choosing the optimal $E_{\text{nom}}$ and $C_{\text{rate}}$ from the given discrete options.}

\subsubsection{Parameters}
\begin{tabular}{@{}lll}
\toprule
Symbol & Description & Unit \\
\midrule
$c_{\text{kWh}}$ & Specific investment cost per unit of energy capacity & EUR/kWh \\
$c_{\text{kW}}$ & Specific investment cost per unit of power capacity & EUR/kW \\
$c_{\text{O\&M}}$ & Annual O\&M cost as a fraction of initial investment & - \\
$r$ & Annual discount rate (WACC) & - \\
$SOH_{\text{init}}$ & Initial State of Health (assumed to be 1) & - \\
\bottomrule
\end{tabular}

\subsubsection{State Dynamics: State of Health (SOH) Evolution}
The profitability in any given year depends on the battery's health. The SOH evolves from one year to the next based on the degradation incurred during the previous year's operation.

\begin{equation}
SOH_{y+1} = SOH_y - \Delta SOH_y(\mathbf{x}, SOH_y) \qquad \forall y \in \{1, ..., 9\}
\end{equation}
\begin{equation}
SOH_1 = SOH_{\text{init}}
\end{equation}

Where $\Delta SOH_y$ is the total degradation (from cyclic and calendar aging) in year $y$. This value is a complex output derived from the operational model, as the optimal dispatch strategy itself depends on the battery's current health.

\subsubsection{Cash Flow Components}

\paragraph{Initial Investment Cost (CAPEX):}
The upfront cost of the BESS, assumed to be a linear function of its size.
\begin{equation}
C_{\text{inv}}(E_{\text{nom}}, P_{\text{max}}) = c_{\text{kWh}} \cdot E_{\text{nom}} + c_{\text{kW}} \cdot P_{\text{max}}
\end{equation}
\textit{Based on the project description, $c_{\text{kWh}} = 200$ EUR/kWh. The cost for power capacity is assumed to be bundled or zero for this analysis.}

\paragraph{Annual Revenue:}
The expected annual revenue $\mathbb{E}[\Pi_y]$ is the most complex term. It represents the outcome of running the detailed operational model for a full year, given a BESS with capacity $E_{\text{nom}} \cdot SOH_y$ and power $P_{\text{max}}$.
\begin{equation}
\mathbb{E}[\Pi_y(\mathbf{x}, SOH_y)] = \text{OperationalModel}(E_{\text{nom}} \cdot SOH_y, P_{\text{max}}, \text{prices}_y)
\end{equation}
For practical implementation, this function is typically replaced by a surrogate model or a simulation using representative days/weeks, which is then scaled to an annual figure.

\paragraph{Annual O\&M Cost:}
A recurring annual cost, typically modeled as a fixed percentage of the initial investment.
\begin{equation}
C_{\text{O\&M},y}(\mathbf{x}) = c_{\text{O\&M}} \cdot C_{\text{inv}}(\mathbf{x})
\end{equation}


\section{Methodology for Solving}
The investment optimization problem is a complex, non-linear, and potentially non-convex problem due to the intricate relationship between investment size, operational strategy, degradation, and revenue. A direct analytical solution is not feasible.

The solution approach is a simulation-based evaluation:
\begin{enumerate}
    \item \textbf{Define a set of candidate investments:} Create a discrete set of investment options $(\mathbf{x}_1, \mathbf{x}_2, ..., \mathbf{x}_N)$ based on the permissible configurations (e.g., different C-rates and energy capacities).
    \item \textbf{For each candidate investment $\mathbf{x}_i$:}
    \begin{enumerate}
        \item Initialize $SOH_1 = 1$.
        \item \textbf{Loop for each year $y=1$ to 10:}
        \begin{itemize}
            \item Run the operational optimization model (e.g., using a rolling-horizon approach for the full year) with the current effective capacity ($E_{\text{nom}} \cdot SOH_y$) to calculate the annual revenue $\Pi_y$ and the annual degradation $\Delta SOH_y$.
            \item Update the state of health: $SOH_{y+1} = SOH_y - \Delta SOH_y$.
        \end{itemize}
        \item \textbf{Calculate NPV:} Use the sequence of generated annual revenues $(\Pi_1, ..., \Pi_{10})$ to compute the total 10-year NPV for the candidate investment $\mathbf{x}_i$.
    \end{enumerate}
    \item \textbf{Select the optimum:} The optimal investment $\mathbf{x}^*$ is the candidate with the highest calculated NPV.
\end{enumerate}
This approach effectively decouples the high-level investment decision from the year-by-year operational decisions, while still capturing the critical long-term dynamics of degradation and discounted cash flows.


\subsection{Motivation: Solving the "aFRR Energy Trap"}
\label{sec:motivation}
Our initial analysis of the \texttt{metadata.json} reveals a critical challenge: aFRR Energy prices are frequently higher than DA prices, but their activation is highly uncertain (e.g., \texttt{HU\_Pos\_zero\_pct} is 59.29\%).

A deterministic model (like the one in \texttt{p2\_bi\_model\_ggdp.tex} \cite{ggdp_model}) will be "tricked" by these high prices. It will naively bid all its capacity into the aFRR Energy market, expecting 100\% activation. This leads to two failures:
\begin{enumerate}
    \item \textbf{Profit Overestimation:} The model's expected profit will be wildly inaccurate, as most bids will not be activated (resulting in 0 EUR revenue).
    \item \textbf{Degradation Miscalculation:} The model will incorrectly calculate its own degradation, as the *physical* cycling does not occur.
\end{enumerate}
We \emph{must} account for this uncertainty.

\subsection{Explanation: The "Expected Value" (EV) Approach}
\label{sec:explanation}
We reject computationally infeasible methods like multi-scenario SAA (Stochastic Optimization) or complex Robust Optimization (RO) with duality \cite{kazemi2017}.

Instead, we implement a highly efficient \textbf{"Expected Value" (EV) model}. This approach is computationally "free" (it adds no variables or constraints) and perfectly solves the "aFRR trap".

We introduce a new set of \textbf{parameters} (not variables), $w_t$, derived directly from the \texttt{metadata.json}:
\begin{itemize}
    \item $w^{\mathrm{pos}}_t = 1.0 - \texttt{afrr\_energy.[Country]\_Pos\_zero\_pct}(t)$
    \item $w^{\mathrm{neg}}_t = 1.0 - \texttt{afrr\_energy.[Country]\_Neg\_zero\_pct}(t)$
\end{itemize}
These parameters ($w_t \in [0, 1]$) represent the historical probability of aFRR energy activation at time $t$. We then multiply our aFRR bids by these probabilities.

This ensures the optimizer makes a rational economic choice:
\begin{itemize}
    \item \textbf{Old Model (Naive):} $\text{Profit}_{aFRR} (100) > \text{Profit}_{DA} (80)$? $\to$ Choose aFRR.
    \item \textbf{New Model (EV):} $\text{Profit}_{aFRR} (100) \times w_t(0.5) < \text{Profit}_{DA} (80)$? $\to$ Correctly choose DA.
\end{itemize}

\subsection{Model (ii): Model (i) + Expected-Value Cyclic Aging Cost}
\label{sec:model-ii-xu}
The logic for the cyclic aging cost, based on Xu et al. (2017) \cite{xu2017}, remains unchanged. The core idea is still to penalize deeper cycles more heavily.
\begin{itemize}
    \item \textbf{Core Variables:} $e_{\mathrm{soc},j}(t)$ (energy in segment $j$), $p^{\mathrm{ch}}_{j}(t)$ (charge to segment $j$), and $p^{\mathrm{dis}}_{j}(t)$ (discharge from segment $j$).
    \item \textbf{Cost Function:} $C^{\mathrm{cyc}} = \sum_{t \in T} \sum_{j \in J} \left( c^{\mathrm{cost}}_{j} \cdot \frac{p^{\mathrm{dis}}_{j}(t)}{\eta_{\mathrm{dis}}} \cdot \Delta t \right)$
    \item \textbf{Logic:} The optimizer will still prefer to discharge from the "cheapest" (shallowest) segment $j$ first.
\end{itemize}
The "magic" is that this cost function is now fed the \textbf{expected physical power}, not the naive 100\% activation power. This is achieved by two minor, "surgical" changes in the \textbf{Objective Function} and the \textbf{Power Balance} constraints.

\subsubsection{Surgical Changes to Implement the EV Model}
To implement this, we modify two key areas of the full model:

\paragraph{Update to Objective Function (Eq. 24 in \texttt{.tex})}
The aFRR Energy profit ($\mathbb{P}^{aFRR\_E}$) is now calculated based on its *expected* revenue.
\begin{equation}
\mathbb{P}^{aFRR\_E} = \sum_{t\in T} \left( \frac{P^{\mathrm{pos}}_{E}(t)}{1000}\, (p^{\mathrm{pos}}_{E}(t) \cdot \mathbf{w_t^{\mathrm{pos}}}) - \frac{P^{\mathrm{neg}}_{E}(t)}{1000}\, (p^{\mathrm{neg}}_{E}(t) \cdot \mathbf{w_t^{\mathrm{neg}}}) \right)\, \Delta t
\end{equation}


\begin{thebibliography}{9}
\bibitem{collath2023}
N. Collath, M. Cornejo, V. Engwerth, H. Hesse, and A. Jossen. (2023). "Increasing the lifetime profitability of battery energy storage systems through aging aware operation." \textit{Applied Energy}, 348, 121531.

\bibitem{huawei2025}
Gen Li. (2025). "Huawei TechArena 2025: BESS Energy Management System." \textit{Internal Project Document}.
\end{thebibliography}

\end{document}
